\documentclass[conference]{IEEEtran}
\IEEEoverridecommandlockouts
% The preceding line is only needed to identify funding in the first footnote. If that is unneeded, please comment it out.
\usepackage{cite}
\usepackage{amsmath,amssymb,amsfonts}
\usepackage{algorithmic}
\usepackage{graphicx}
\usepackage{textcomp}
\usepackage{xcolor}
\def\BibTeX{{\rm B\kern-.05em{\sc i\kern-.025em b}\kern-.08em
    T\kern-.1667em\lower.7ex\hbox{E}\kern-.125emX}}
\begin{document}

\title{Image to Text Recognition for Detecting Human and Machine Altered News in Social Media
} 
\author{\IEEEauthorblockN{Abdullah Kamal}
\IEEEauthorblockA{\textit{Department of Computer Science} \\
\textit{Texas State University, San Marcos, TX} \\
aai27@txstate.edu}
\and
\IEEEauthorblockN{Zaid Jamal}
\IEEEauthorblockA{\textit{Department of Computer Science} \\
\textit{Texas State University, San Marcos, TX} \\
z\_j33@txstate.edu}
\and
\IEEEauthorblockN{Gabriel Rosales}
\IEEEauthorblockA{\textit{Department of Computer Science} \\
\textit{Texas State University, San Marcos, TX} \\
gmr94@txstate.edu}
\and
\IEEEauthorblockN{Brian Robinson}
\IEEEauthorblockA{\textit{Department of Computer Science} \\
\textit{Texas State University, San Marcos, TX} \\
bwr37@txstate.edu}
\and
\IEEEauthorblockN{Zachary Sotny}
\IEEEauthorblockA{\textit{Department of Computer Science} \\
\textit{Texas State University, San Marcos, TX} \\
zachsotny@txstate.edu}
\and
\IEEEauthorblockN{Heena Rathore}
\IEEEauthorblockA{\textit{Department of Computer Science} \\
\textit{Texas State University, San Marcos, TX} \\
heena.rathore@txstate.edu}
}
\maketitle

\begin{abstract}
Fake news on social media platforms has become a pressing issue, undermining the reliability of factual information. Platforms like Twitter, Instagram, and Reddit lack measures to recognize falsified information that is circulated via images. To address this challenge, we propose a solution that utilizes the Optical Character Recognition (OCR) capabilities of the Google Cloud Vision API to extract text from news images. The extracted text is then cross-referenced with the New York Times (NYT) database to verify the authenticity of the news articles. Our testing on human-altered, fake, and real news images yielded a 91\% accuracy in detecting falsified news articles. This paper offers a promising approach to combat misinformation in image-based news content shared on social media platforms, thereby contributing to the preservation of factual information integrity
\end{abstract}
\begin{IEEEkeywords}
social media, neural network, API, image-to-text, optical character recognition, Application Programming Interface
\end{IEEEkeywords}

\section{Introduction}
The proliferation of fake news on social media has emerged as a formidable challenge for individuals seeking reliable information via their most frequented applications. A study conducted by Pew Research revealed that approximately 50\% of their sample population relied on social media as their primary source of obtaining news \cite{pew2021news}. Despite this trend, social media companies have been complacent in combating fake news on their sites, as noted in an article from Reuters \cite{reuters}. Screenshots of altered or completely fictitious articles are not uncommon on social media, presenting a distinct problem for social media companies trying to eliminate fake news. Since fake news is usually thought to be in the form of social media posts and not images, this problem presents a unique issue in the spreading of fake news \cite{acmpaper}.

According to a study done in 2019, \cite{fakenews} deep learning neural networks can be very conclusive in their ability to validate news found on social media cites like Twitter. While current models such as the ones proposed in that study can accurately determine whether a string of text has been generated using a neural language model, they still face challenges in identifying human-made alterations to news and detecting fake news conveyed through images \cite{paper4}. To address this issues, some works have been proposed to address the propagation of human alterations of news. Kai Shu introduces a novel approach to fake news detection by combining linguistic and visual cues extracted from news articles and accompanying images \cite{paper1}. They analyze textual content, meta-data, and visual features to identify instances of fake news. Similarly, M. Aldwairi utilized natural language processing techniques and sentiment analysis to classify articles as fake or genuine, while also examining the influence of social media interactions on news veracity \cite{paper2}. Y. Liu addresses the detection of misinformation in online social networks through a graph-based approach, leveraging network structure and propagation patterns to detect false information. Their experiments utilize real-world data sets obtained from Twitter \cite{paper3}. While the solutions proposed in these papers are impressive, they fail to address the aforementioned issue of fake news contained within images of articles that contain human or machine alterations.

Building upon these research works, our paper proposes a novel approach that leverages OCR to extract data from images sourced from a dataset comprising approximately 50 manually generated and altered articles sourced from the NY Times. These images are then validated by cross-referencing them with existing news articles. By comparing the textual content extracted from these images with articles from reputable sources like the NYT, we can assess the authenticity of news in any image posted to social media. Our solution presents a comprehensive data set of images associated with news articles, which is utilized to train and evaluate our proposed image-to-text recognition model for fake news detection. Our approach achieves a near 100\% accuracy rate in detecting fake news in images, contributing significantly to the prevention of the spread of misinformation. To enhance the understanding and effectiveness of fake news detection, our research focuses on integrating multiple modalities, specifically combining textual and visual analysis. This integration allows us to categorize news from social media based on its authenticity, distinguishing between news that can be corroborated from reliable sources and news that cannot be verified \cite{git}.

\section{Methods}

\subsection{Image to Text Recognition}

Our approach capitalizes on the advanced image recognition capabilities of the Google Cloud Vision \emph{Application Programming Interface} (API), combined with its OCR techniques. By leveraging this API, we are able to analyze the content shared by users on various social media platforms, specifically focusing on images associated with news articles. The OCR functionality of this API allows us to extract the textual content embedded within these images. Once the image is provided to the Google Cloud Vision API, it employs OCR techniques to convert the textual information within the image into machine-readable text. The API returns the extracted text in a structured JavaScript Object Notation response format, allowing us to further process and analyze the content.

\subsection{Database Analysis}

To assess the authenticity of the news content extracted from the image, we utilize the extracted text in a search query within the NYT database, leveraging their Developers API. The purpose of this step is to cross-reference the extracted text with the content available in the NY Times database. If a match is found, indicating that the news article or its content is present in the database returned through the search query, it classifies the image content as real news. Conversely, if no matching content is found, it raises an alert on the client side indicating the potential presence of fake news.

\subsection{User Interface}

In order to provide an accessible and user-friendly solution, we have developed a dynamic webpage that complements our image recognition framework. The website serves as the interface for users to submit images for content validation, utilizing the aforementioned image recognition and fake news detection capabilities. Users can conveniently upload an image, and the system automatically processes it through the Google Cloud Vision API and cross-references the extracted text with the NYT database.

\section{Results}

\begin{figure}
\centering
    \includegraphics[width=0.4\textwidth]{grover.png}
    \caption{Percentage Accuracy of Article Detection}
\label{fig:fakenews1}
\end{figure}

In the evaluation of our solution, we tested a total of 52 news articles, consisting of 23 real articles, 15 fake articles consisting of articles generated by both ChatGPT and humans, and 14 altered articles. Through cross-referencing with the existing news articles in the NYT database, our solution achieved a remarkable accuracy of 100\% in detecting both real and fake news as shown in Figure~\ref{fig:fakenews1}. It further demonstrated an accuracy rate of 71\% in identifying modified news articles, properly identifying 10 out of 14 articles that underwent human editing ranging from minor changes in wording to the replacement of entire paragraphs. Calculating precision and recall scores of 0.92 and 1.0, we acquired an F1 score of 0.96 and a cumulative accuracy of 91\%, demonstrating our solution's viability for detecting fake news. Despite this, there were certain limitations due to the Google API, including the inability to filter out extraneous text not part of an article and the inability to process lengthy articles. These limitations necessitate edge case improvements for enhanced accuracy and scalability.

We intend to expand to incorporate multiple news outlets, which can enhance the accuracy and authenticity of the extracted textual content. Additionally, addressing the limitations related to text filtering and processing of longer articles will be crucial for the practical implementation of our solution. Overall, the results obtained in our evaluation highlight its potential as a valuable tool in combating the spread of fake news in image-based news content on social media.

\section{Conclusion}

In this paper, we propose the development of a user-friendly website that seamlessly facilitates the validation of image accuracy on social media. By leveraging the OCR capabilities of the Google Cloud Vision API and cross-referencing with the New York Times database, our technical approach offers an effective method for detecting fake news within images. These findings demonstrate the robustness of our solution, with a perfect detection rate for real and fake news, and a commendable 93\% accuracy in identifying modified news articles. Although improvements are needed to address text filtering and recognition of longer articles, our proof-of-concept testing establishes our solution as a promising tool to enhance the accuracy of fake news detection, holding significant potential for practical deployment in combating the spread of misinformation in social media.

\begin{thebibliography}{00}
\bibitem{pew2021news} 
Pew Research Center, ``News consumption across social media in 2021''. \url{https://www.pewresearch.org/journalism/2021/09/20/news-consumption-across-social-media-in-2021/} [accessed on April 9, 2023].

\bibitem{reuters}
Reuters. (2023, February 7). Big Tech not doing enough to remove fake news, activist NGO Avaaz says. Retrieved from \url{https://www.reuters.com/technology/big-tech-not-doing-enough-remove-fake-news-activist-ngo-avaaz-says-2023-02-07/}

\bibitem{acmpaper}
Shu, K., Mahudeswaran, D., Wang, S., Lee, D., & Liu, H. (2020). FakeNewsNet: A Data Repository with News Content, Social Context, and Spatiotemporal Information for Studying Fake News on Social Media. Big Data, 8(3), 171-188. doi:10.1089/big.2020.0062. PMID: 32491943. Retrieved from https://doi.org/10.1089/big.2020.0062.

\bibitem{fakenews}
Abdullah-All-Tanvir, E. M. Mahir, S. Akhter and M. R. Huq, "Detecting Fake News using Machine Learning and Deep Learning Algorithms," 2019 7th International Conference on Smart Computing & Communications (ICSCC), Sarawak, Malaysia, 2019, pp. 1-5, doi: 10.1109/ICSCC.2019.8843612.

\bibitem{paper4}
B. Narwal, "Fake News in Digital Media," 2018 International Conference on Advances in Computing, Communication Control and Networking (ICACCCN), Greater Noida, India, 2018, pp. 977-981, doi: 10.1109/ICACCCN.2018.8748586.

\bibitem{paper1}
K. Shu et al., ``Fake News Detection on Social Media: A Data Mining Perspective''. \textit{SIGKDD Explor. Newsl}. vol. 19, no. 1, pp. 22–36. Retrieved from \url{https://doi.org/10.1145/3137597.3137600}, 2017.
\bibitem{paper2}
Aldwairi, M., & Alwahedi, A. (2018).
\emph{Detecting Fake News in Social Media Networks}
Procedia Computer Science, 141, 215-222. DOI: \href{https://doi.org/10.1016/j.procs.2018.10.171}{10.1016/j.procs.2018.10.171}

\bibitem{paper3}
Y. Liu and Y.-F. Wu, “Early Detection of Fake News on Social Media Through Propagation Path Classification with Recurrent and Convolutional Networks”, AAAI, vol. 32, no. 1, Apr. 2018.

\bibitem{git}
Zaid323918. (n.d.). Fake-News-Image-Recognition (Version 1.0). Available at: \url{https://github.com/Zaid323918/Fake-News-Image-Recognition/tree/main} (Accessed: May 23, 2023).

\end{thebibliography}
\vspace{12pt}

\end{document}
